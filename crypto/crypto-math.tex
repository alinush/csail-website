\documentclass[12pt]{article}
\usepackage[margin=1in]{geometry} % set the margins to 1in on all sides
\usepackage{verbatim}   % \begin{verbatim}
\usepackage{fullpage}
\usepackage{biblatex}   % \printbibliography
\usepackage{hyperref}   % creates links in table of contents
\usepackage{parskip}    % do not indent paragraphs
\usepackage{amsmath,amsthm,amssymb,amsfonts,amscd}
\usepackage{mathabx}    % \divides, \notdivides
\usepackage{graphics}
%\usepackage{graphicx}
%\usepackage{subfigure}

% various theorems, numbered by section
\newtheorem{thm}{Theorem}[section]
\newtheorem{lem}[thm]{Lemma}
\newtheorem{prop}[thm]{Proposition}
\newtheorem{cor}[thm]{Corollary}
\newtheorem{conj}[thm]{Conjecture}
\newtheorem{definition}[thm]{Definition}

\newcommand{\Zp}{\mathbb{Z}^{\ast}_p}
\newcommand{\Zq}{\mathbb{Z}^{\ast}_q}
\newcommand{\Zn}{\mathbb{Z}^{\ast}_n}
\newcommand{\Znz}{\mathbb{Z}/n\mathbb{Z}}
\newcommand{\Qp}{\mathbb{Q}_p}
\newcommand{\Qn}{\mathbb{Q}_n}
\newcommand{\G}{\mathbb{G}}
\newcommand{\gen}[1]{\langle #1 \rangle}
%\newcommand{\divides}{\bigm|}
\newcommand{\note}{\emph{Note:} }
\newcommand{\todo}{\textbf{TODO:} }
\newcommand{\example}{\emph{Example:} }
\newcommand{\sz}[1]{\left|#1\right|}

% fixes spacing before a theorem/definition with the amsthm package,
% which gets messed up when the parskip is used
\makeatletter
\def\thm@space@setup{%
  \thm@preskip=\parskip \thm@postskip=0pt
  }
\makeatother

\bibliography{crypto-math}

\begin{document}

\title{Crypto math}
\author{Alin Tomescu\\
  \texttt{alinush@mit.edu}}

\maketitle

\begin{abstract}
A summary of the mathematical concepts used in cryptography. This text is inspired from and aims to extend Ron Rivest's second lecture on group theory in 6.857. First, we begin with an introduction to groups and the discrete log problem. Next, we introduce the Diffie-Hellman key-exchange protocol and the active attacks against it.
\end{abstract}

\tableofcontents

\newpage

\section{Group theory review}

Here, we are talking about multiplicative groups (where the operation between
group elements is something \emph{resembling} multiplication)

\begin{definition}
$(\G, \cdot)$ is a \emph{finite abelian group} of size $t$ if:
\begin{itemize}
  \item $\exists$ identity $1$ such that $\forall a \in \G, a\cdot 1 = 1\cdot a = a$
  \item $\forall a     \in \G, \exists b \in \G$ such that $a\cdot b = 1$
  \item $\forall a,b,c \in \G, a\cdot (b\cdot c) = (a\cdot b)\cdot c$
  \item $\forall a,b   \in \G, a\cdot b = b\cdot a$
\end{itemize}
\end{definition}

\subsection{Order (cardinality) of a group and group generators}

\begin{definition}
The \emph{order} of $a$ in $\G$ is denoted by $order(a)$ and is equal to the least $u$ such that $a^u = 1$
\end{definition}

\begin{thm}[Lagrange]
In a finite abelian group of size $t$, for all $a \in \G$, $order(a) \divides t$
\end{thm}

\begin{thm}
In a finite abelian group of size $t$, $\forall a \in \G, a^t = 1$
\end{thm}

\begin{example}
$a^{(p-1)} = 1, \forall a \in \Zp$ because $|\Zp| = 1$
\end{example}

\begin{definition}
$\gen{a} = \{a^i : i \ge 0\} = $ subgroup generated by $a$.
\end{definition}

\begin{definition}
If $\gen{a} = \G$ then $\G$ is \emph{cyclic} and $a$ is a
\emph{generator} of $\G$.
\end{definition}

\begin{note}
$|\gen{a}| = order(a)$
\end{note}

\emph{Exercise:} In a finite abelian group $\G$ of order $t$, where $t$ is
prime, we have: $\forall a \in \G$, if $a \ne 1 \Rightarrow a$ is a generator of $\G$.

\emph{Solution:} We know that the size of any subgroup of $\G$ must divide $t$.
Since $t$ is prime, any subgroup can either have size $1$ or $t$. Thus, only
trivial subgroups can exist: the subgroup made up of the identity element
($\{1\}$) and $\G$ itself. Since $a \ne 1$, any subgroup generated by $a$ cannot
be equal to $\{1\}$ because it will have to contain $a$ itself which is
different than $1$.  Thus, if $a$ generates any subgroup, it has to generate
$\G$ itself. How do we know that $a$ generates any subgroup at all then? We know
$a \in G \Rightarrow a^u \in G, \forall u$ and, informally, we know that there
cannot be a $u, 1 < u < t$ such that $a^u = 1$ because that would create a
subgroup of $\G$ of size $u$, which would imply $u \divides t$, which would be
false since $t$ is prime.

\begin{thm}
$\Zp$ is always cyclic (i.e. there exists a generator within $\Zp$)
\end{thm}


\subsection{Discrete logs}

\begin{thm}
If $\G$ is a cyclic group of order $t$ and generator $g$ then the relation
$x \leftrightarrow g^x$ is one-to-one between $[0, 1, \dots, t-1]$ and $\G$.
\end{thm}

\begin{align*}
x \mapsto g^x & \text{ : exponentiation, "powering-up"}\\
g^x \mapsto x & \text{ : discrete logarithm (DL)}
\end{align*}

Computing \emph{discrete logarithms} (the DL problem) is commonly assumed to be
hard/infeasible for well-chosen groups $\G$ (e.g. $\Zp$ for $p$ a large randomly
chosen prime).

\todo Notes about various DL attacks. See the WeakDH paper?

\subsection{API for a group}

In practice, we need to be able to translate \emph{bits of data} or
\emph{messages} from a message space $M$ to group elements of $\G$. We need an
one-to-one (injective) function $f : M
\rightarrow \G$ such that $f(m) \in \G$ can be chosen to represent message $m
\in M$.

\todo Does $f$ need to be onto (surjective) as well? Are there cases
where we need to reverse $f$?

\begin{example}
If we have $\Zp$ with $p > 2^k$, then we can represent any $k$-bit
message $m$ as a number $x \in [0, 2^k)$ because if $x \in [0, 2^k)$, then $x
\in \Zp$.
\end{example}

\begin{note}
For some groups, finding an easily-computable, space-efficient $f$ may
be a little hard.
\end{note}

Typically, any library that implements a group should provide the following calls:

\begin{center}
\begin{tabular}{ c | l | c }
Operation & API call & Comments \\
\hline\hline
creation & $\G \leftarrow createGroup(\dots)$ &\\
identity & $\G.identity()$ &\\
random element & $x \leftarrow \G.random()$ &\\
product & $x \cdot y$ & or $+$\\
inverse & $x^{-1}$ & or $-x$\\
power & $x^k, x \in \G, k \in \mathbb{Z}$ & or $k\cdot x$\\
size & $\G.order()$ & $|G|$, not always implemented\\
list & $\G.elements()$ & not always implemented\\
represent & $x \leftarrow G.rep(m)$ &\\
unrepresent & $m \leftarrow G.unrep(x)$ &\\
generator & $\G.generator()$ &\\
discrete log & $x \leftarrow \G.discreteLog(g, y)$ & s.t $g^x = y$, not always
\emph{efficiently} possible \\
\end{tabular}
\end{center}

\section{Diffie-Hellman (DH) key exchange (1976)}

How can we establish a shared secret in the presence of a passive eavesdropper
Eve?

Let $\G$ be a cyclic group with generator $g$. $\G$ and $g$ are fixed and public.

Alice and Bob can agree on a shared secret key $k$ as follows:

\begin{enumerate}
  \item Alice chooses secret $x$ randomly from $[0, \dots, |\G|-1]$. Note that
  $x \notin \G$.
  \item Alice computes $ g^x$ as her \emph{public key}. Note that $g^x \in \G$
  and Alice is the only one who knows $x$, the \emph{discrete log} of $g^x$.
  \item Bob, similarly picks a $y$ and computes $g^y$.
  \item Alice and Bob exchange $g^y$ and $g^x$. Eve sees them.
  \item Assuming discrete logs are hard to compute, Eve cannot learn neither $x$
  nor $y$ because she will have a hard time computing the discrete log of $g^x$ or
  $g^y$.
  \item Alice computes $k = (g^y)^x = g^{xy}$
  \item Bob computes (the same) $k = (g^x)^y = g^{xy}$
  \item Alice and Bob have agreed on a shared key $k$
\end{enumerate}

\subsection{Computation Diffie-Hellman (CDH) assumption}
Can Eve compute $g^{xy}$ from $g^x$ and $g^y$? We assume she can't and
refer to this assumption as the \emph{Computational Diffie-Hellman} (CDH)
assumption.

\begin{thm}
CDH is hard $\Rightarrow$ Diffie-Hellman key exchange is secure (i.e.
Eve does not learn $k$)
\end{thm}

\subsection{Active attacks against DH}

\begin{note}
Can use $k$ to encrypt and/or MAC messages.
\end{note}

\begin{note}
If not using an authenticated encryption mode like EAX, derive separate keys
for encryption and authentication $k_{enc} = PRF(k, enc)$ and $k_{mac} = PRF(k, mac)$.
\end{note}

\begin{note}
$g^x$ and $g^y$ are assumed to be the right public keys for Alice and Bob.
\end{note}

What if Eve is active and changes them on their way to Alice and Bob?

If Eve is \emph{active}, she can perform a \emph{man-in-the-middle attack}:

\begin{enumerate}
  \item Alice sends $g^x$ to Bob, but Eve replaces it with $g^e$, for which she
  knows $e$. Eve records $g^x$.
  \item Bob sends $g^y$ to Bob, but Eve replaces it with $g^v$, for which she
  knows $v$. Eve records $g^y$.
  \item Alice got $g^v$ from Eve, so she will compute shared key $k_1 = g^{xv}$
  \item Bob got $g^e$ from Eve, so she will compute shared key $k_2 = g^{ye}$
  \item Alice and Bob think they are talking to each other, but they agreed to
  different keys.
  \item Eve can compute $k_1 = g^{xv}$ herself: she knows $v$
  \item Eve can compute $k_2 = g^{ye}$ herself: she knows $e$
  \item When Alice sends a message to Bob, encrypted and/or MACd with $k_1$, Eve
  can decrypt and tamper with the message and then reencrypt and MAC it under $k_2$
  for Bob.
  \item Eve can do the same for Bob's messages to Alice.
\end{enumerate}

To fix this problem, we need to prevent Eve from swapping Alice and Bob's public
keys on the wire. One solution is to have a \emph{certification authority} (CA)
digitally sign $g^x$ and $g^y$ so that Eve cannot replace them.

\begin{note}
Still not perfect. What if Eve colludes with the CA?
\end{note}

\begin{note}
What if Eve has friends with public keys signed by the CA. Those friends can maybe give Eve their private keys and Eve could still pull the attack $\Rightarrow$ the
digital signature has to \emph{cryptographically bind} the user's identity (Alice)
to her public keys $g^x$. This way, if Eve replaces Alice's public key with
her friend's Diana public key, Bob will detect this when he verifies the signature
on the public key: the signature will not verify against Alice's name.
\end{note}

\section{The five groups}

\subsection{Common groups and their notations}

$\Zp*$: The multiplicative group of integers modulo a prime $p$

$\Zn*$: The multiplicative group of integers modulo a non-prime $n$ that are coprime to $n$

$\mathbb{Z}[x]$: The additive group of polynomials in one variable denoted by
$x$ and with coefficients in $\mathbb{Z}$

$\Znz$: The additive (or multiplicative, without $0$) group of integers
modulo $n$. If $n$ is prime, then $\Znz$ forms a field. Otherwise, it's just
a ring.

$\Zn$

$\Zp$

$\Qp$

$\Qn$

$\text{GF}(l)$: A finite field or a Galois field is a field that contains a
finite number $l$ of elements. An example is the prime field $\text{GF}(p)$ (or
$\pmb{\text{Z}}/p\pmb{\text{Z}}$, $\mathbb{F}_p$ or $\pmb{\text{F}}_p$) of order (size) $p$. This is the field of integers from $0, 1, \dots, p-1$.

\subsection{$\Zp$}
\begin{definition}
$\Zp = \{ a : 1 \le a < p\}$, where $p$ is prime
\end{definition}

$\Zp$ is always cyclic (i.e. has a generator). There are non-constructive proofs
for this.

If $p = 2q + 1$ and $q$ is prime, then $p$ is a \emph{safe prime} and half of
$\Zp$ elements are generators and the other half are squares $\Qp$.

\todo Proof?

\subsection{$\Qp$, quadratic residues (squares) mod prime $p$}
\begin{definition}
$\Qp = \{a^2 : 1 \le a < p\} \subsetneq \Zp$
\end{definition}

\todo Is $a < p$ or $a^2 < p$?

\begin{thm}
$|\Qp| = \frac{1}{2}\sz{\Zp} = \frac{(p-1)}{2}$
\end{thm}

\begin{thm}
$\Qp$ is cyclic: If $\gen{g} = \Zp$, then $\gen{g^2} = \Qp$
\end{thm}

Thus, $\Qp = \{g^{2i} : 0 \le i < \frac{p-1}{2}\}$, if $\gen{g} = \Zp$

If $p = 2q + 1$, then $\sz{\Qp} = \frac{(p-1)}{2} = q$ and \emph{any element} of
$\Qp$ (other than 1) generates $\Qp$. To find a generator, take the square of
any element $a \in \Zp - \{1, p-1\}$

\todo Proof for why $\Zp$ is split that way? Is $p-1$ the only generator
that would generate a subgroup of size $2$? The identity (i.e. $1$) would generate
the subgroup of size $1$, and apparently all squares generate $\Qp$ of size $q$,
which means the rest either generate $\Zp$ or the subgroup of size $2$.

\subsection{$\Zn$}

\begin{definition}
$\Zn = \{a : \gcd(a, n) = 1, \text{where } 1 \le a < n\}$
\end{definition}

\begin{definition}
$\sz{\Zn} = \phi(n)$, the \emph{totient} function.
\end{definition}

If $n = pq$ where $p,q$ are distinct odd primes, then $\Zn$ is \emph{not} cyclic.

...but the \emph{Chinese Remainder Theorem} says there exists an isomorphism
from $\Zn$ to $\Zp \times \Zq$.

\subsection{$\Qn$, quadratic residues (squares) mod (non-prime) $n$}
\begin{definition}
$\Qn = \{a^2 : 1 \le a < n, \text{where } \gcd(a, n) = 1\}$
\end{definition}

\todo Is $a < p$ or $a^2 < p$?

\todo Is $\gcd(a, n) = 1$ or $\gcd(a^2, n) = 1$?

\begin{thm}
If $n = pq$ where $p = 2r+1$ and $q = 2s+1$, then $\sz{\Qn} = rs$ and
$\Qn$ is cyclic.
\end{thm}

\subsection{TODO: Elliptic curves}

\nocite{*}
\printbibliography

\end{document}
